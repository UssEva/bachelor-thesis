% !TeX spellcheck = en_GB
% !TeX encoding = UTF-8
% !TeX root = ../thesis.tex
\chapter{Background}\label{chap:background}
As this thesis merges known concepts from \scratch{} analysis and \ngram{}, we introduce the concepts our work is based on in the following sections. Section~\ref{sec:analysing-scratch} introduces \scratch{} and its functionality as a block-based programming language as well as the \scratch{} analyser \litterbox{}. Section~\ref{sec:language-models} introduces the concept of a \ngram{}. It is necessary to understand the difference between n-grams and a n-gram model and the way of using it in order to obtain valuable information about \scratch{} code.


\section{Analysing \scratch{} Programs}\label{sec:analysing-scratch}

% \textit{blocks}\footnote{\url{https://en.scratch-wiki.info/wiki/Blocks}, last accessed April 29, 2020}
% \hyperref[def:code-smells]{code smells}
\subsection{The Structure of \scratch{}}\label{subsec:scratch}
\scratch{}\footnote{\url{https://en.scratch-wiki.info/wiki/Scratch}, last accessed August 19, 2020} is a block-based programming language designed for kids that was created by the Massachusetts Institute of Technology (MIT). With over 50 million registered users and shared projects\footnote{\url{https://scratch.mit.edu/statistics/}, last accessed August 19, 2020} it is one of the most popular tools for teaching children from a young age how to develop computational thinking skills to solve problems programmatically.

The 'drag-and-drop-programming' style that \scratch{} is utilizing allows the programmer to pick from a pallet of existing blocks and build scripts by combining these code pieces like a jigsaw puzzle.
\textit{Blocks}\footnote{\url{https://en.scratch-wiki.info/wiki/Blocks}, last accessed August 19, 2020} are separated into different types that include \textit{hat, stack, reporter, boolean and cap}. Each data type has a specific shape that indicates their intended usage in order to avoid errors in the syntax of the code. In contrast to text-based programming languages, the \textit{blocks} help children to memorize the commands and avoid structure errors that otherwise would very likely occur in the beginning. 

After combining different \textit{blocks}, \textit{scripts}\footnote{\url{https://en.scratch-wiki.info/wiki/Script}, last accessed August 19, 2020} are created which resemble code methods in \java{}. \textit{Scripts} are found in the code of the actors that perform the actions, called \textit{sprites}\footnote{\url{https://en.scratch-wiki.info/wiki/Sprite}, last accessed August 19, 2020} as well as the \textit{stage}\footnote{\url{https://en.scratch-wiki.info/wiki/Stage}, last accessed August 19, 2020} which visualises the background of the created program. 

% TODO Add example code (script with picture)

\subsection{The Static \scratch{} Code Analyser \litterbox{}}\label{subsec:litterbox}
\litterbox{}\footnote{\url{https://gitlab.infosun.fim.uni-passau.de/se2/litterbox}, last accessed August 19, 2020} is a static code analysis tool for detecting bugs in \scratch{} projects. The occurrence of bugs in code is mostly the consequence of recurring bad code writing habits that visualise in code smells and refactoring opportunities. With the help of \litterbox{} these bug patterns can be filtered and analysed in order to fix found bugs and avoid similar mistakes in the future. \litterbox{} is developed at the Chair of Software Engineering II and the Didactics of Informatics of the University of Passau.

% TODO Add access information for LitterBox


\section{N-gram Language Model}\label{sec:language-models}
% TODO Introduce language models

\subsection{Definition of N-grams}\label{subsec:ngram}
% TODO Explain Markov chain and probability calculation

\subsection{Probability Calculation with N-gram Model}\label{subsec:ngram}
% TODO Define key words (Token, GramSize, SequenceLength, ReportingSize, Minimum Token Occurence, Probability Threshold





