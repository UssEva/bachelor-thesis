% !TeX spellcheck = en_GB
% !TeX encoding = UTF-8
% !TeX root = ../thesis.tex
\chapter{Background}\label{chap:background}
As this thesis merges known concepts from \scratch{} analysis and \ngram{}, we introduce the concepts our work is based on in the following sections. Section~\ref{sec:analysing-scratch} introduces \scratch{} and its functionality as a block-based programming language. Section~\ref{sec:language-models} introduces the concept of a \ngram{}. It is necessary to understand the difference between n-grams and a n-gram model and the way of using it in order to obtain valuable information about the given code.


\section{Analysing Scratch Programs}\label{sec:analysing-scratch}

% \textit{blocks}\footnote{\url{https://en.scratch-wiki.info/wiki/Blocks}, last accessed April 29, 2020}
% \hyperref[def:code-smells]{code smells}
% \textit{actors}
\subsection{Basic Elements of \scratch}\label{subsec:scratch}
% TODO Explain Scratch structure

\subsection{The Static \scratch{} Code Analyser \litterbox}\label{subsec:litterbox}
% TODO Litterbox


\section{N-gram Language Models}\label{sec:language-models}
% TODO Introduce language models

\subsection{Definition of N-gram}\label{subsec:ngram}

\subsection{Model Structure}\label{subsec:ngram}


