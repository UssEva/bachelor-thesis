% !TeX spellcheck = en_GB
% !TeX encoding = UTF-8
% !TeX root = ../thesis.tex

\newcommand{\numlarge}{75,277}
\newcommand{\monthstart}{December 2019}
\newcommand{\monthend}{January 2020}
\newcommand{\parsingexcp}{114}
\newcommand{\successfullyanalysed}{74,914}
\newcommand{\calculatedngrams}{138,299}
\newcommand{\creationtime}{17 days}

\chapter{Evaluation}\label{chap:evaluation}

The main research points for this bachelor's thesis are answers to the following questions:
\begin{itemize}
\item[\textbf{RQ1}] How effective are \ngram{s} for bug detection in comparison to \litterbox{}?
\item[\textbf{RQ2}] What kinds of violations were detected by the \ngram{}?
\item[\textbf{RQ3}] Does the \ngram\ work in the case of open task solutions?
\item[\textbf{RQ4}] Efficiency of general \ngram\ vs project-specific model?
\end{itemize}


\section{Data Sets}\label{sec:dataset}
This section shows the two different data sets that were used for the bachelor's thesis, specifically, for the training of the model like it is introduced in Subsection~\ref{subsec:trainingset} and for bug detection which Subsection~\ref{subsec:bugset} focuses on.

\subsection{Big Data Set}\label{subsec:trainingset}
In order to have a sufficient number of sequences to calculate a probability distribution from, we decided to build the \ngram{} on a large data set. The dataset consists of \numlarge\ \scratch\ projects. From \monthstart\ to \monthend\ we downloaded the most recent projects with the \scratch\ REST API\footnote{\url{https://github.com/LLK/scratch-rest-api/wiki}, last accessed May 8, 2020}. We did exclude remixes from the data set. \litterbox\ could not parse \parsingexcp\ projects, thus the \ngram\ was created of \successfullyanalysed\ projects without any exceptions and consists of \calculatedngrams\ calculated n-grams. The creation of the model took \creationtime\ and was conducted on machines equipped with Intel Xeon E5-2650 v2 @ 2.60 GHz CPUs with 256 GiB of RAM.

\subsection{Pupil's Projects}\label{subsec:bugset}
The second data set we used for creating project-specific models is a set of correct as well as defective solutions of five small coding tasks for students. Based on the task the pupils had to implement, we call these sets \textit{Monkey}, \textit{Elephant}, \textit{Cat}, \textit{Horse} and \textit{Fruit Catching} task. We used solutions of pupils which originate in primary programming education~\cite{katharina} for the Monkey, Elephant, Cat and Horse tasks. For the Fruit Catching task we used the same data set as Stahlbauer et al. in their work about testing \scratch\ programs automatically~\cite{whisker}. In addition, a set of open task solutions that pupil's could experiment with, was used for Section~\ref{sec:open}. Table~\ref{tab:small-dataset} shows the number of projects for each task. For bug detection we chose one representative project of each task which are listed in Table~\ref{tab:buggy-projects} and analysed them with \litterbox\ and the created n-gram models. The creation of the project-specific models as well as bug detection with the pupil's projects was done in a few seconds for every set of solutions and did not throw any exceptions. All experiments with the small data sets were conducted on a Swift SF314-57 with an Intel i5 core and 8 GB RAM.
 
\begin{table}[H]
    \centering
    \caption[Projects of small data set]{\label{tab:small-dataset}Projects of small data set for project-specific models}
    \begin{tabular}{lr}
        \toprule
        Task & \#Projects \\
        \midrule
        Fruit Catching & 42 \\
        Monkey & 120 \\
        Elephant & 130 \\
        Cat & 129 \\
        Horse & 73 \\
        Open & 295 \\
        \bottomrule
    \end{tabular}
\end{table}

\begin{table}[H]
    \centering
    \caption[Representative projects for each task]{\label{tab:buggy-projects}Representative projects for each task used for bug detection}
    \begin{tabular}{lrrrr}
        \toprule
        Task & Project & Size & \#Sprites & \#Blocks\\
        \midrule
        Fruit Catching & K7\_S02.sb3 & 176 KB & 3.0 & 81.0\\
        Monkey & ID\_065\_Aufgabe-Affenjagd.sb3 & 181 KB & 3.0 & 18.0 \\
        Elephant & ID\_118\_Aufgabe-Elefant.sb3 & 456 KB & 4.0 & 23.0 \\
        Cat & ID\_005\_Aufgabe-Katze.sb3 & 86 KB & 2.0 & 14.0 \\
        Horse & ID\_080\_Aufgabe-Pferd.sb3 & 6 KB & 1.0 & 14.0 \\
        Open & m043.sb3 & 304 KB & 3.0 & 26.0 \\ 
        \bottomrule
    \end{tabular}
\end{table}


\section{RQ1: Comparison to Litterbox}\label{sec:litterbox}
After setting the \hyperref[def:gram_size]{\textit{gram size}} to 3 and \hyperref[def:sequence_length]{\textit{sequence lengths}} to 2 for the first analysis and 3 for the second bug detection analysis, project-specific models for each pupil task from Table~\ref{tab:small-dataset} are used for a comparison between \litterbox{} and the \ngram{} approach. The same projects from Table~\ref{tab:buggy-projects} are analysed by \litterbox{} and the \ngram{} to test how many sequences are reported after knowing the failed tests of each project through \whisker{}. The \hyperref[def:reporting_size]{\textit{reporting size}} is unlimited in this case. Table~\ref{tab:litterbox} shows the number of reported smells or sequences of each method. 

\begin{table}[H]
    \centering
    \caption[The number of reported bugs found by \litterbox{} and N-gram model]{\label{tab:litterbox}The number of reported bugs found by \litterbox{} and N-gram model}
    \begin{tabular}{lrrr}
        \toprule
        Task & \#FailedTests & \#LitterboxSmells & \#NgramSequences \\
        \midrule
        Fruit Catching & - & 113 & 42 \\
        Monkey & 0 & 2 & 8 \\
        Elephant & 0 & 1 & 5 \\
        Cat & 0 & 0 & 6 \\
        Horse & 1 & 0 & 8 \\
        \bottomrule
    \end{tabular}
\end{table}

The analysis shows that even though in some cases \litterbox\ did not find smells and \whisker\ tests all passed, the \ngram\ reported sequences that pointed out potential bugs or unusual use cases. This shows that the \ngram\ has a different range of violations it can detect in \scratch\ projects.

\section{RQ2: Violation Classification}\label{sec:violations}
The analysis procedure is continued by analysing specifically the bugs that are reported by the \ngram{}. For each task one project is manually analysed to estimate the rate of false positives. The \hyperref[def:minimum_token_occurrence]{\textit{minimum token occurrence}} varies from the size of the model. Table~\ref{tab:buggy-projects} shows the projects that were chosen for further assessment as well as their further information. After all by \ngram{} detected potential bugs were collected in the candidate bug set, the defective code is manually classified into the following categories: \textit{True Bugs}, \textit{Unusual Use Cases} or \textit{False Positives}. Table~\ref{tab:violations} displays the numbers for each category.

\begin{table}[H]
    \centering
    \caption[The categorization of all reported bugs]{\label{tab:violations}The categorization of all reported bugs}
    \begin{tabular}{lrrrrr}
        \toprule
        Task & \parbox[t]{2.2cm}{Probability\\Threshold} & \#Reported & \#True & \#UnusualUse & \#FalsePositive \\
        \midrule
        Fruit Catching & 0.6\% & 23 & 15 & 3 & 5 \\
        Monkey & 1.6\% & 3 & 2 & 1 & 0 \\
        Elephant & 1.6\% & 1 & 0 & 1 & 0 \\
        Cat & 1.6\% & 1 & 0 & 1 & 0 \\
        Horse & 1.6\% & 4 & 3 & 0 & 1 \\
        \bottomrule
    \end{tabular}
\end{table}

If a sequence is an unusual use case, it will not be detected by \litterbox\ because it is not a code smell. Most of the time a unusual use is a consequence of an extension of the original task and therefore not part of the usual used sequences that are needed for a project to pass the test.

When we compare the code smells that \litterbox\ is able to find with the bugs that can be detected by the \ngram\, then we see that the \ngram\ is not able to detect long script or unused variable in the \scratch\ code. But the \ngram\ is also capable of finding dead code, empty scripts as well as empty bodies in if-else statements. 
 

\section{RQ3: Open Solutions}\label{sec:gram_size}
For a wider perspective on the usage of the \ngram\ approach to bug detection we selected a set of the pupil's solutions where they could try out all the things they had newly learned. This means that there is no unique solution for these tasks, just open interpretations of simple projects. Therefore there is no method to determine if a reported sequence is an unusual use case. The project-specific model, we created out of the solutions, reported the following bugs displayed in Table~\ref{tab:open} for a set of example projects of the data set. In this analysis the \hyperref[def:minimum_token_occurrence]{\textit{minimum token occurrence}} is set to 0.5\%. 

\begin{table}[H]
    \centering
    \caption[Reported bugs of open projects]{\label{tab:open}Reported bugs of open projects}
    \begin{tabular}{lrrr}
        \toprule
        Task & \#Reported & \#True & \#FalsePositive \\
        \midrule
        m043.sb3 & 4 & 3 & 1 \\   
        m052-Zauberei.sb3 & 2 & 1 & 1 \\ 
        m067-Apfelsuche.sb3 & 1 & 0 & 1 \\
        m074-Sternenjaeger.sb3 & 1 & 0 & 1 \\
        m091 2.sb3 & 4 & 0 & 4 \\
        w002-Weltraumangriff.sb3 & 2 & 2 & 0 \\            
        \bottomrule
    \end{tabular}
\end{table}

The frequent occurrence of false positives could be a consequence of the missing unusual use case categorisation. A lot of sequences just were not used as often by the novice programmers and when they appear one time their probability is consequently pretty low. 


\section{RQ4: General vs Project-specific Model}\label{sec:project-specific}
Instead of just one model that is based on a large data set, we used many smaller models that are specific to the projects we wanted to analyse. For each pupil's task, we created a model out of the given solutions and its reference solution. This way the model has specific data that is related to its task.

The bigger model may have a wider range of potential n-grams but this does not correlate with better results. Like it is shown in Table~\ref{tab:versus} with the Cat task as an example, the reported probabilities of the general model are not usable in a way to identify potential bugs, code smells or unusual use cases in the \scratch\ code in comparison to project-specific models. 

One reason are negative possibilities that appeared in the reports. The origin of the negative calculations are still unclear and have to be analysed in a more thorough way in the near future. A programming mistake or miscalculations could be the reason of the negative numbers. But even in the case of a correct probability estimation, project-specific n-gram models are still more accurate and reliable in their usage.  

\begin{table}[H]
    \centering
    \caption[General vs project-specific model]{\label{tab:versus}General vs project-specific model}
    \begin{tabular}{lrr}
        \toprule
        Sequence & General & Project-specific \\
        \midrule
        GreenFlag, RepeatForeverStmt & -163.7176695714846 & 0.0052368762629570725 \\
        Touching, SayForSecs & -93.4693467460925 & 0.0025830339970531286 \\
        PointTowards, MoveSteps & -56.557244922395064 & 3.548409418312439E-4 \\
        NumberLiteral, IfThenStmt & 72.48561394813275 & 0.0033964044507544403 \\
        NumberLiteral & 72.48561394813275 & 2.3721578534715566 \\
        StringLiteral, NumberLiteral & -46.75499727794927 & 0.0027517806291449814 \\
        \bottomrule
    \end{tabular}
\end{table}


\section{Threats to Validity}\label{sec:threats-to-validity}
There is no guarantee that this bachelor's thesis is free of faults or miscalculations. Here are a few points that should be addressed about the here described and executed research as well as evaluation of the \ngram{} on \scratch{} code.

\paragraph{Implementation of N-gram Model.}
To compare the results of \litterbox{} with the n-gram model approach, we implemented the n-gram language model as close as possible to the information given from the \bugram{}~\cite{bugram} paper and based on the \AST{} that is created by \litterbox{} out of \scratch{} projects in JSON format. For our implementation, we have tried our best to tune the configuration parameters to obtain the best results. Our comparison is fair because both \litterbox{} as well as \ngram{} are evaluated on the same projects.

\paragraph{Bugs are manually verified.}
Following the \bugram{}~\cite{bugram} paper, reported bugs were assessed manually in order to approve and classify all low probability \hyperref[def:token]{\textit{token}} sequences and distinguish between, i.e., true bugs, refactoring opportunities and false positives. Since we are not the original programmers of the analysed projects that are featured in this bachelor's thesis, the examination of the code is not objective. But because of common practice, it is an acceptable approach for the assessment of the given code.